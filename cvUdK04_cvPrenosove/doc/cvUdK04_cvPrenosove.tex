\documentclass[a4paper, 10pt, ]{article}

\input{misc/preamble.tex}

\def\oznacenieCasti{cvUdK04 - ZS2019}





\begin{document}





\fontsize{12pt}{22pt}\selectfont

\centerline{\textsf{Úvod do kybernetiky} \hfill \textsf{\oznacenieCasti}}

\fontsize{18pt}{22pt}\selectfont





\begin{flushleft}
    \textbf{\textsf{Cvičenie štvrté}}
\end{flushleft}





\normalsize

\bigskip

% \tableofcontents

\bigskip

\vspace{18pt}



\section{Úlohy cvičenia}

\begin{enumerate}

	\item Vypočítajte póly lineárnych dynamických systémov daných prenosovými funkciami.

	\item Nakreslite prechodové charakteristiky lineárnych dynamických systémov daných prenosovými funkciami.

	\item Nakreslite frekvenčné charakteristiky lineárnych dynamických systémov daných prenosovými funkciami. Frekvenčné charakteristiky znázornite ako Bodeho charakteristiky a ako Nyquistove charakteristiky.

\end{enumerate}



\noindent
Lineárne dynamické systémy sú pre toto cvičenie definované prenosovou funkciou so všeobecnými parametrami v tvare
\begin{equation}
	G(s) = \frac{b_2 s^2 + b_1 s + b_0}{a_3 s^3 + a_2 s^2 + a_1 s + a_0} e^{-Ds}
\end{equation}
a tabuľkou, v ktorej sú uvedené hodnoty parametrov jednotlivých systémov:



\bigskip




\begin{centering}
\catcode`\-=12

%\caption{Názov tabuľky}
%\label{tabulka}

\centering

\begin{tabular*}{1.0\columnwidth}{ @{\extracolsep{\fill}} c  c c c c c c c c c c c}
\toprule
\multicolumn{1}{l}{Systém} & \multicolumn{8}{l}{Parameter} & \multicolumn{2}{l}{Obrázok}   \\
 & $b_2$ & $b_1$ & $b_0$ & $a_3$ & $a_2$ & $a_1$ & $a_0$ & $D$ & \multicolumn{1}{c}{PCH} & \multicolumn{1}{c}{FCH}   \\
 %&  &  &  &  &  &  &  &  & \multicolumn{1}{l}{PCH} & \multicolumn{1}{l}{FCH}   \\
\midrule
%1 & & & & & & & & & & \\
$1$. & & & $1$ & & & $1$ & $1$ & & \multirow{4}{*}[-6pt]{\rotatebox{90}{Obr. 1.}} & \multirow{1}{*}[0pt]{\rotatebox{90}{6.}}  \\ \cmidrule{11-11}
$2$. & & & $1$ & & & $1$ & $1$ & $5$ & & \multirow{1}{*}[0pt]{\rotatebox{90}{7.}}  \\ \cmidrule{11-11}
$3$. & & & $0,1$ & & & $1$ & $0$ & & & \multirow{1}{*}[0pt]{\rotatebox{90}{8.}}  \\ \cmidrule{11-11}
$4$. & & & $0,1$ & & & $1$ & $0$ & $3$ & & \multirow{1}{*}[0pt]{\rotatebox{90}{9.}}  \\ \cmidrule{10-11}
$5$. & & $1$ & $1$ & & & $3$ & $1$ & & \multirow{2}{*}[0pt]{\rotatebox{90}{2.}} & \multirow{2}{*}[0pt]{\rotatebox{90}{10.}} \\
$6$. & & $1$ & $-1$ & & & $3$ & $1$ & & & \\ \cmidrule{10-11}
$7$. & & & $0,5$ & & $1$ & $2$ & $1$ & &  \multirow{4}{*}[0pt]{\rotatebox{90}{Obr. 3.}} & \multirow{4}{*}[0pt]{\rotatebox{90}{Obr. 11.}} \\
$8$. & & & $0,5$ & & $1$ & $1$ & $1$ & & & \\
$9$. & & & $0,5$ & & $1$ & $0,2$ & $1$ & & & \\
$10$. & & & $0,5$ & & $1$ & $0$ & $1$ & &  & \\ \cmidrule{10-11}
$11$. & & & $0,2$ & & $1$ & $1$ & $0$ & & \multirow{3}{*}[-3pt]{\rotatebox{90}{Obr. 4.}} & \multirow{1}{*}[1pt]{\rotatebox{90}{12.}}  \\ \cmidrule{11-11}
$12$. & & & $0,2$ & & $1$ & $0$ & $0$ & & & \multirow{1}{*}[1pt]{\rotatebox{90}{13.}}  \\ \cmidrule{11-11}
$13$. & & & $0,2$ & & $1$ & $0$ & $0$ & $4$ & & \multirow{1}{*}[1pt]{\rotatebox{90}{14.}}  \\ \cmidrule{10-11}
$14$. & $1$ & $2$ & $2$ & $1$ & $0,3$ & $4,03$ & $0,401$ &  & \multirow{2}{*}[0pt]{\rotatebox{90}{5.}} & \multirow{1}{*}[1pt]{\rotatebox{90}{15.}}  \\ \cmidrule{11-11}
$15$. & $1$ & $2$ & $2$ & $1$ & $0,3$ & $4,03$ & $0,401$ & $6$ & & \multirow{1}{*}[1pt]{\rotatebox{90}{16.}}  \\
\bottomrule
\end{tabular*}



\end{centering}



%\end{table}

\bigskip

\noindent
Tabuľka určuje aj číslo obrázka, do ktorého nakreslite príslušnú charakteristiku (PCH prípadne FCH). Niektoré charakteristiky sú na spoločnom obrázku.



























\end{document}
