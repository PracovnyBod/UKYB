\documentclass[a4paper, 10pt, ]{article}

\input{misc/preamble.tex}

\def\oznacenieCasti{cvUdK05\_zadanie - ZS2020}


\definecolor{mynotecolor}{gray}{0.33}


\begin{document}





\fontsize{12pt}{22pt}\selectfont

\centerline{\textsf{Úvod do kybernetiky} \hfill \textsf{\oznacenieCasti}}

\fontsize{18pt}{22pt}\selectfont





\begin{flushleft}
    \textbf{\textsf{Zadanie prvé}}
\end{flushleft}





\vspace{18pt}

\fontsize{14pt}{18pt}\selectfont

\begin{flushleft}
    Prevodová charakteristika - simulovaný systém
\end{flushleft}




\normalsize

% \bigskip

% \tableofcontents

% \bigskip

% \vspace{18pt}


\noindent
Zadanie spolu za 10 bodov.

O práci na úlohách je potrebné referovať písomne formou krátkej správy (referátu). Referát/dokument sa odovzdáva do AIS. Pre termín odovzdania pozri príslušné miesto odovzdania v AIS.





\paragraph{Úlohy}


\medskip

\begin{enumerate}[leftmargin=0pt, labelsep=4mm, itemsep=0pt]


	\item Opis predmetného systému.
    \begin{itemize}[leftmargin=0pt, labelsep=4mm, itemsep=0pt] \color{mynotecolor}
        \item Systém, ktorým sa tu zaoberáme je kyvadlo, tak ako bolo predstavené v~predchádzajúcich týždňoch.
        \item Opíšte systém, ktorého prevodová charakteristika sa bude zisťovať: diferenciálna rovnica, označenie veličín/signálov, fyzikálne jednotky (prípadne aj obrázok) \hfill {\color{MidnightBlue} (1b)}
	\end{itemize}



    \item Vzorová numerická simulácia
    \begin{itemize}[leftmargin=0pt, labelsep=4mm, itemsep=0pt] \color{mynotecolor}
        \item Zostavte numerickú simuláciu (simulačnú schému), pomocou ktorej bude možné predviesť rôzne situácie (simulované): nenulové začiatočné podmienky, nenulový vstupný signál (kombinácie).
        \item Proces zostavenia numerickej simulácie stručne zdokumentujte tak, aby ho bolo možné jednoznačne zrekonštruovať - aby čitateľ dokumentácie vedel zostaviť rovnakú numerickú simuláciu. \hfill {\color{MidnightBlue} (0,5b)}
        \item Realizujte vzorové numerické simulácie tak aby ste prezentovali funkčnosť zostavenej numerickej simulácie. \hfill {\color{MidnightBlue} (0,5b)}
	\end{itemize}


    \item Prevodová charakteristika - získanie „surových“ dát \hfill {\color{MidnightBlue} (1b)}
    \begin{itemize}[leftmargin=0pt, labelsep=4mm, itemsep=0pt] \color{mynotecolor}
        \item Táto časť sa týka získania východiskového súboru dát/údajov.
        \item Dokumentácia uvažovaných rozsahov vstupného a výstupného signálu.
        \item Stanovenie hodnôt vstupného signálu, pre ktoré sa budú merať hodnoty výstupného signálu.
        \item Opis realizácie získania jednotlivých ustálených hodnôt výstupného signálu (možnosti napríklad: jeden beh a viac ustálených hodnôt (schodíky), alebo viac behov a~v~každom jedna ustálená hodnota).
	\end{itemize}


    \item Spracovanie „surových“ dát  \hfill {\color{MidnightBlue} (1,4b)}
    \begin{itemize}[leftmargin=0pt, labelsep=4mm, itemsep=0pt] \color{mynotecolor}
        \item Opis postupu, pomocou ktorého sa z východiskového súboru dát získajú jednotlivé body prevodovej charakteristiky.
	\end{itemize}

    \item Prevodová charakteristika - výsledky \hfill {\color{MidnightBlue} (5,6b)}
    \begin{itemize}[leftmargin=0pt, labelsep=4mm, itemsep=0pt] \color{mynotecolor}
        \item Grafické a tabuľkové znázornenie prevodovej charakteristiky.
	\end{itemize}








\end{enumerate}




\end{document}
