\documentclass[a4paper, 10pt, ]{article}

\usepackage[slovak]{babel}

\usepackage[utf8]{inputenc}
\usepackage[T1]{fontenc}

\usepackage[left=4cm,
			right=4cm,
			top=2.1cm,
			bottom=2.6cm,
			footskip=7.5mm,
			twoside,
			marginparwidth=3.5cm,
			%showframe,
			]{geometry}

\usepackage{graphicx}
\usepackage[dvipsnames]{xcolor}

% ------------------------------

\usepackage{lmodern}

\usepackage[tt={oldstyle=false,proportional=true,monowidth}]{cfr-lm}

% ------------------------------

\usepackage{amsmath}
\usepackage{amssymb}
\usepackage{amsthm}

\usepackage{booktabs}
\usepackage{multirow}
\usepackage{array}
\usepackage{dcolumn}

\usepackage{natbib}

\usepackage[singlelinecheck=true]{subfig}


% ------------------------------


\usepackage{sectsty}
\allsectionsfont{\sffamily}


\usepackage{titlesec}
\titleformat{\paragraph}[hang]{\sffamily  \bfseries}{}{0pt}{}
\titlespacing*{\paragraph}{0mm}{3mm}{1mm}


\usepackage{fancyhdr}
\fancypagestyle{plain}{%
\fancyhf{} % clear all header and footer fields
\fancyfoot[C]{\sffamily {\bfseries \thepage}\ | {\scriptsize\oznacenieCasti}}
\renewcommand{\headrulewidth}{0pt}
\renewcommand{\footrulewidth}{0pt}}
\pagestyle{plain}


% ------------------------------


\makeatletter

	\def\@seccntformat#1{\protect\makebox[0pt][r]{\csname the#1\endcsname\hspace{5mm}}}

	\def\cleardoublepage{\clearpage\if@twoside \ifodd\c@page\else
	\hbox{}
	\vspace*{\fill}
	\begin{center}
	\phantom{}
	\end{center}
	\vspace{\fill}
	\thispagestyle{empty}
	\newpage
	\if@twocolumn\hbox{}\newpage\fi\fi\fi}

	\newcommand\figcaption{\def\@captype{figure}\caption}
	\newcommand\tabcaption{\def\@captype{table}\caption}

\makeatother


% ------------------------------


\def\naT{\mathsf{T}}

\hyphenpenalty=6000
\tolerance=6000


% ------------------------------


\usepackage[pdfauthor={},
			pdftitle={},
			pdfsubject={},
			pdfkeywords={},
			% hidelinks,
			colorlinks=true,
			breaklinks,
			]{hyperref}






% ------------------------------

\usepackage{enumitem}





\usepackage[titles]{tocloft}

\setlength{\cftsecindent}{-12mm}
\setlength{\cftsecnumwidth}{12mm}
\renewcommand{\cftsecpresnum}{\hfill}
\renewcommand{\cftsecaftersnum}{\hspace{4mm}}

\setlength{\cftsubsecindent}{-12mm}
\setlength{\cftsubsecnumwidth}{16mm} % 12 + 4
\renewcommand{\cftsubsecpresnum}{\hfill}
\renewcommand{\cftsubsecaftersnum}{\hspace{8mm}} % 4 + 4 mm

\setlength{\cftsubsubsecindent}{-12mm}
\setlength{\cftsubsubsecnumwidth}{20mm} % 12 + 4 + 4
\renewcommand{\cftsubsubsecpresnum}{\hfill}
\renewcommand{\cftsubsubsecaftersnum}{\hspace{12mm}} % 4 + 4 + 4 mm

\renewcommand{\cftsecpagefont}{\lstyle \bfseries}
\renewcommand{\cftsubsecpagefont}{\lstyle}
\renewcommand{\cftsubsubsecpagefont}{\lstyle}



\setlength{\cftparaindent}{-16mm}
\setlength{\cftparanumwidth}{28mm} % 16 + 4 + 4 + 4
\renewcommand{\cftparapresnum}{\hfill}
\renewcommand{\cftparaaftersnum}{\hspace{16mm}} % 4 + 4 + 4 + 4 mm

% ------------------------------









\usepackage{listings}



\renewcommand{\lstlistingname}{Výpis kódu}
\renewcommand{\lstlistlistingname}{Výpisy kódu}




%New colors defined below
\definecolor{codegreen}{rgb}{0,0.6,0}
\definecolor{codegray}{rgb}{0.5,0.5,0.5}
\definecolor{codepurple}{rgb}{0.58,0,0.82}
\definecolor{backcolour}{rgb}{0.95,0.95,0.95}

%Code listing style named "mystyle"
\lstdefinestyle{mystyle}{
  backgroundcolor=\color{backcolour},
  commentstyle=\fontfamily{lmtt}\fontsize{8.5pt}{8.75pt}\selectfont\color{codegreen},
  keywordstyle=\fontfamily{lmtt}\fontsize{8.5pt}{8.75pt}\selectfont\bfseries\color{Blue},
  stringstyle=\fontfamily{lmtt}\fontsize{8.5pt}{8.75pt}\selectfont\color{codepurple},
  basicstyle=\fontfamily{lmtt}\fontsize{8.5pt}{8.75pt}\selectfont,
  breakatwhitespace=false,
  breaklines=true,
  captionpos=t,
  keepspaces=true,
  numbers=left,
  numbersep=4mm,
  numberstyle=\fontfamily{lmtt}\fontsize{8.5pt}{8.75pt}\selectfont\color{lightgray},
  showspaces=false,
  showstringspaces=false,
  showtabs=false,
  tabsize=2,
  % xleftmargin=10pt,
  framesep=10pt,
  language=Python,
  escapechar=|,
}




% ------------------------------



\graphicspath{{fig/}{../../extObr/}}


\def\oznacenieCasti{cvUdK05\_zadanie - ZS2020}


\definecolor{mynotecolor}{gray}{0.33}


\begin{document}





\fontsize{12pt}{22pt}\selectfont

\centerline{\textsf{Úvod do kybernetiky} \hfill \textsf{\oznacenieCasti}}

\fontsize{18pt}{22pt}\selectfont





\begin{flushleft}
    \textbf{\textsf{Zadanie prvé}}
\end{flushleft}





\vspace{18pt}

\fontsize{14pt}{18pt}\selectfont

\begin{flushleft}
    Prevodová charakteristika - simulovaný systém
\end{flushleft}




\normalsize

% \bigskip

% \tableofcontents

% \bigskip

% \vspace{18pt}


\noindent
Zadanie spolu za 10 bodov.

O práci na úlohách je potrebné referovať písomne formou krátkej správy (referátu). Referát/dokument sa odovzdáva do AIS. Pre termín odovzdania pozri príslušné miesto odovzdania v AIS.





\paragraph{Úlohy}


\medskip

\begin{enumerate}[leftmargin=0pt, labelsep=4mm, itemsep=0pt]


	\item Opis predmetného systému.
    \begin{itemize}[leftmargin=0pt, labelsep=4mm, itemsep=0pt] \color{mynotecolor}
        \item Systém, ktorým sa tu zaoberáme je kyvadlo, tak ako bolo predstavené v~predchádzajúcich týždňoch.
        \item Opíšte systém, ktorého prevodová charakteristika sa bude zisťovať: diferenciálna rovnica, označenie veličín/signálov, fyzikálne jednotky (prípadne aj obrázok) \hfill {\color{MidnightBlue} (1b)}
	\end{itemize}



    \item Vzorová numerická simulácia
    \begin{itemize}[leftmargin=0pt, labelsep=4mm, itemsep=0pt] \color{mynotecolor}
        \item Zostavte numerickú simuláciu (simulačnú schému), pomocou ktorej bude možné predviesť rôzne situácie (simulované): nenulové začiatočné podmienky, nenulový vstupný signál (kombinácie).
        \item Proces zostavenia numerickej simulácie stručne zdokumentujte tak, aby ho bolo možné jednoznačne zrekonštruovať - aby čitateľ dokumentácie vedel zostaviť rovnakú numerickú simuláciu. \hfill {\color{MidnightBlue} (0,5b)}
        \item Realizujte vzorové numerické simulácie tak aby ste prezentovali funkčnosť zostavenej numerickej simulácie. \hfill {\color{MidnightBlue} (0,5b)}
	\end{itemize}


    \item Prevodová charakteristika - získanie „surových“ dát \hfill {\color{MidnightBlue} (1b)}
    \begin{itemize}[leftmargin=0pt, labelsep=4mm, itemsep=0pt] \color{mynotecolor}
        \item Táto časť sa týka získania východiskového súboru dát/údajov.
        \item Dokumentácia uvažovaných rozsahov vstupného a výstupného signálu.
        \item Stanovenie hodnôt vstupného signálu, pre ktoré sa budú merať hodnoty výstupného signálu.
        \item Opis realizácie získania jednotlivých ustálených hodnôt výstupného signálu (možnosti napríklad: jeden beh a viac ustálených hodnôt (schodíky), alebo viac behov a~v~každom jedna ustálená hodnota).
	\end{itemize}


    \item Spracovanie „surových“ dát  \hfill {\color{MidnightBlue} (1,4b)}
    \begin{itemize}[leftmargin=0pt, labelsep=4mm, itemsep=0pt] \color{mynotecolor}
        \item Opis postupu, pomocou ktorého sa z východiskového súboru dát získajú jednotlivé body prevodovej charakteristiky.
	\end{itemize}

    \item Prevodová charakteristika - výsledky \hfill {\color{MidnightBlue} (5,6b)}
    \begin{itemize}[leftmargin=0pt, labelsep=4mm, itemsep=0pt] \color{mynotecolor}
        \item Grafické a tabuľkové znázornenie prevodovej charakteristiky.
	\end{itemize}








\end{enumerate}




\end{document}
