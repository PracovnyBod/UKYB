\documentclass[a4paper,10pt]{article}
\usepackage[utf8]{inputenc}
\usepackage[slovak]{babel}
\usepackage{amsmath}
\usepackage{graphicx}

\begin{document}

\title{Riešenie lineárnej diferenciálnej rovnice druhého rádu s využitím Laplaceovej transformácie}
\author{Martin Dodek}
\pagestyle{plain}
\maketitle

\section{Diferenciálna rovnica}
Majme diferenciálnu rovnicu druhého rádu s konštantnými koeficientami (rovnaká ako na minulom cvičení):

\begin{equation}
\label{eq:diff_eq}
\ddot{y}+a_1\dot{y}+a_0y=0
\end{equation}

Riešenie diferenciálnej rovnice spočíva v nájdení \emph{funkcie} $f(t)$ popisujúcej vývoj premennej $y$ v čase.
\begin{equation*}
	y(t)=f(t)
\end{equation*}

Počiatočné podmienky (hodnoty premenných a ich derivácii v čase $t=0$) nech sú dané ako $y_0$ a $\dot{y}_0$.

Laplaceovu transformáciu je možné použiť iba pre riešenie \underline{lineárnych} diferenciálnych rovníc (tento prípad).

\section{Laplaceova transformácia}
Laplaceova transformácia je integrálnou transformáciou definovanou ako:
\begin{equation}
\label{eq:LPT}
 \mathcal{L}\left\lbrace f(t)\right\rbrace=\int_{0}^{\infty}{e^{-st}f(t)dt}
\end{equation}
Interpretovať ju môžeme ako integrál vlastných kmitov, respektíve ako koreláciu signálu $f(t)$ (v časovej oblasti) s komplexnou exponenciálou $e^{-st}$ (v $s$ oblasti).

Kde $s$ je komplexná premenná a je označovaná aj ako Laplaceov operátor.

\pagebreak

\subsection{Vzťah Laplaceovej transformácie a derivácie}
Z pohľadu kybernetiky je najdôležitejšou vlastnosťou jej vzťah k derivácii signálu $f(t)$.
Zjednodušene povedané, operátor $s$ reprezentuje operáciu časovej derivácie.
Presnejšie:
\begin{equation}
\label{eq:LPT prva derivacia}
	\mathcal{L}\left\lbrace\dot{f}(t) \right\rbrace=s\mathcal{L}\left\lbrace f(t)\right\rbrace-f(0)
\end{equation}

Kde $f(0)$ je počiatočná podmienka - hodnota signálu v čase 0.

Pre vyššie derivácie signálu potom platí reťazové pravidlo.

Druhá derivácia:
\begin{equation}
\label{eq:LPT druha derivacia}
	\mathcal{L}\left\lbrace\ddot{f}(t) \right\rbrace=s\left(  s\mathcal{L}\left\lbrace f(t)\right\rbrace-f(0) \right)-\dot{f}(0)
\end{equation}

Teda po úprave:
\begin{equation}
\label{eq:LPT nta derivacia}
	\mathcal{L}\left\lbrace\ddot{f}(t) \right\rbrace=s^2 \mathcal{L}\left\lbrace f(t)\right\rbrace -s f(0) -\dot{f}(0)
\end{equation}

Kde $f(0)$ je počiatočná hodnota signálu a $\dot{f}(0)$ je počiatočná hodnota jeho prvej derivácie.

Všeobecný vzťah pre obraz derivácie vyššieho stupňa $n$

\begin{equation}
	\mathcal{L}\left\lbrace f^n(t) \right\rbrace=s^n \mathcal{L}\left\lbrace f(t)\right\rbrace - s^{(n-1)}f(0) \ldots -f^{(n-1)}(0)
\end{equation}


\subsection{Laplaceova transformácia exponenciálnej funkcie}
Pre účely riešenia diferenciálnych rovníc a teda aj analýzu lineárnych systémov je zvlášť dôležitý Laplaceov obraz exponenciálnej funkcie.

Jej obraz bude:

\begin{equation*}
\mathcal{L}\left\lbrace e^{at}\right\rbrace=\int_{0}^{\infty}{e^{-st}e^{at}dt}
\end{equation*}

\begin{equation*}
\mathcal{L}\left\lbrace e^{at}\right\rbrace=\int_{0}^{\infty}{e^{(a-s)t}dt}
\end{equation*}


\begin{equation*}
\mathcal{L}\left\lbrace e^{at}\right\rbrace=\frac{1}{a-s}\left[e^{(a-s)t} \right]_{0}^{\infty}
\end{equation*}

Čo vo výsledku je:

\begin{equation}
\label{eq:LPT exp}
\mathcal{L}\left\lbrace e^{at}\right\rbrace=\frac{1}{s-a}
\end{equation}

\pagebreak

\section{Riešenie diferenciálnej rovnice}

Diferenciálnu rovnicu \eqref{eq:diff_eq} môžeme s využitím obrazu Laplaceovej transformácie derivácie signálu $y(t)$ aj jeho vyšších derivácii \eqref{eq:LPT nta derivacia} formulovať ako:

\begin{equation}
\label{eq:LPT diferencialnej rovnice}
Y(s)\left(s^2+a_1s+a_0\right)=sy_0+\dot{y}_0+a_1 y_0
\end{equation}

Vyjadrením obrazu signálu $Y(s)$ získavame racionálnu funkciu:
\begin{equation}
\label{eq:racionalna funkcia}
Y(s)=\frac{sy_0+\dot{y}_0+a_1y_0}{s^2+a_1s+a_0}
\end{equation}

Pre účely riešenia diferenciálnej rovnice (získanie časovej funkcie $y(t)$ ) musíme túto funkciu rozložiť na parciálne zlomky:

\begin{equation}
\label{eq:parcialne zlomky}
Y(s)=\frac{sy_0+\dot{y}_0+a_1 y_0}{(s-\lambda_1)(s-\lambda_2)}=\frac{A}{s-\lambda_1}+\frac{B}{s-\lambda_2}
\end{equation}

Je preto nutné nájsť korene menovateľa racionálnej funkcie : $\lambda_1$ a $\lambda_2$.
Zostavíme teda charakteristikcú rovnicu (kvadratickú rovnicu):
\begin{equation}
 s^2+a_1s+a_0=0
\end{equation}

Riešením charakteristickej rovnice sú korene $\lambda_1$ a $\lambda_2$ (všeobecne komplexné):
\begin{equation}
\label{eq:charakteristická rovnica riešenie}
 \lambda_{1,2}=\frac{-a_1\pm \sqrt{a_1^2-4a_0}}{2}
\end{equation}


Pre určenie koeficentov $A$ a $B$ si pravú stranu rovnice \eqref{eq:parcialne zlomky} upravíme na spoločného menovateľa:

\begin{equation}
\label{eq:parcialne zlomky upravene}
Y(s)=\frac{sy_0+\dot{y}_0+a_1 y_0}{(s-\lambda_1)(s-\lambda_2)}=\frac{A(s-\lambda_2)+B(s-\lambda_1)}{(s-\lambda_1)(s-\lambda_2)}
\end{equation}

Zavedieme rovnosť polynómov:
\begin{equation*}
sy_0+\dot{y}_0+a_1 y_0=(A+B)s-(A\lambda_2+B\lambda_1)
\end{equation*}

Koeficienty $A$ a $B$ potom určíme ako riešenie vzniknutej sústavy lineárnych rovníc.

Pre formuláciu všeobecného riešenia diferenciálnej rovnice využijeme znalosť Laplaceovho obrazu exponenciálnej funkcie \eqref{eq:LPT exp}.

Realizujeme tak vlastne \underline{spätnú} Laplaceovu transformáciu rovnice \eqref{eq:parcialne zlomky}.


\begin{equation}
\label{eq: riešenie všeobecné}
y(t)=Ae^{\lambda_1t}+Be^{\lambda_2t}
\end{equation}

\pagebreak

\section{Príklad}
Majme konkrétnu diferenciálnu rovnicu (rovnaká ako na minulom cvičení):

\begin{equation*}
\ddot{y}+3\dot{y}+2y=0 
\end{equation*}

Počiatočné podmienky nech sú dané:
\begin{equation*}
	y_0=4\quad \dot{y}_0=3;
\end{equation*}

Realizujeme Laplaceovu transformáciu tejto rovnice v zmysle \eqref{eq:LPT diferencialnej rovnice}

\begin{equation*}
	Y(s)\left(s^2+3s+2\right)=4s+3+3\times4
\end{equation*}

Získavame racionálnu funkciu, podobne ako vo všeobecnom riešení \eqref{eq:racionalna funkcia}

\begin{equation*}
Y(s)=\frac{4s+15}{s^2+3s+2}
\end{equation*}

Charakteristická rovnica bude zostavená v tvare:
\begin{equation*}
	s^2+3s+2=0
\end{equation*}

Riešenie charakteristickej rovnice \eqref{eq:charakteristická rovnica riešenie}
\begin{equation*}
	\lambda_{1,2}=\frac{-3\pm\sqrt{9-8}}{2}
\end{equation*}

Korene teda budú reálne:
\begin{equation*}
	\lambda_2=-1 \qquad \lambda_1=-2
\end{equation*}


Racionálnu funkciu rozložíme na parciálne zlomky rovnako ako v \eqref{eq:parcialne zlomky}

\begin{equation*}
Y(s)=\frac{4s+15}{(s+1)(s+2)}=\frac{A}{s+1}+\frac{B}{s+2}
\end{equation*}

Úpravou na spoločného menovateľa \eqref{eq:parcialne zlomky upravene}

\begin{equation*}
Y(s)=\frac{4s+15}{(s+1)(s+2)}=\frac{A(s+2)+B(s+1)}{(s+1)(s+2)}
\end{equation*}

Zo zavedenej rovnosti polynómov vyplýva:

\begin{equation*}
(A+B)s+(2A+B)=4s+15
\end{equation*}

Získame sústavu lineárnych rovníc:
\begin{equation*}
\begin{array}{c}
	A+B=4 \\
	2A+B=15
\end{array}
\end{equation*}

Jej riešenie bude:

\begin{equation*}
A=11 \quad B=-7
\end{equation*}

Využijeme znalosť Laplaceovho obrazu exponenciálnej funkcie \eqref{eq:LPT exp}.
Ak realizujeme spätnú Laplaceovu transformáciu potom riešenie bude v zmysle \eqref{eq: riešenie všeobecné}

\begin{equation*}
y(t)=-7e^{-t}+11e^{-2t}
\end{equation*}


\end{document}